\documentclass[]{article}
\usepackage{lmodern}
\usepackage{amssymb,amsmath}
\usepackage{ifxetex,ifluatex}
\usepackage{fixltx2e} % provides \textsubscript
\ifnum 0\ifxetex 1\fi\ifluatex 1\fi=0 % if pdftex
  \usepackage[T1]{fontenc}
  \usepackage[utf8]{inputenc}
\else % if luatex or xelatex
  \ifxetex
    \usepackage{mathspec}
    \usepackage{xltxtra,xunicode}
  \else
    \usepackage{fontspec}
  \fi
  \defaultfontfeatures{Mapping=tex-text,Scale=MatchLowercase}
  \newcommand{\euro}{€}
\fi
% use upquote if available, for straight quotes in verbatim environments
\IfFileExists{upquote.sty}{\usepackage{upquote}}{}
% use microtype if available
\IfFileExists{microtype.sty}{%
\usepackage{microtype}
\UseMicrotypeSet[protrusion]{basicmath} % disable protrusion for tt fonts
}{}
\usepackage[margin=1in]{geometry}
\usepackage{color}
\usepackage{fancyvrb}
\newcommand{\VerbBar}{|}
\newcommand{\VERB}{\Verb[commandchars=\\\{\}]}
\DefineVerbatimEnvironment{Highlighting}{Verbatim}{commandchars=\\\{\}}
% Add ',fontsize=\small' for more characters per line
\usepackage{framed}
\definecolor{shadecolor}{RGB}{248,248,248}
\newenvironment{Shaded}{\begin{snugshade}}{\end{snugshade}}
\newcommand{\KeywordTok}[1]{\textcolor[rgb]{0.13,0.29,0.53}{\textbf{{#1}}}}
\newcommand{\DataTypeTok}[1]{\textcolor[rgb]{0.13,0.29,0.53}{{#1}}}
\newcommand{\DecValTok}[1]{\textcolor[rgb]{0.00,0.00,0.81}{{#1}}}
\newcommand{\BaseNTok}[1]{\textcolor[rgb]{0.00,0.00,0.81}{{#1}}}
\newcommand{\FloatTok}[1]{\textcolor[rgb]{0.00,0.00,0.81}{{#1}}}
\newcommand{\CharTok}[1]{\textcolor[rgb]{0.31,0.60,0.02}{{#1}}}
\newcommand{\StringTok}[1]{\textcolor[rgb]{0.31,0.60,0.02}{{#1}}}
\newcommand{\CommentTok}[1]{\textcolor[rgb]{0.56,0.35,0.01}{\textit{{#1}}}}
\newcommand{\OtherTok}[1]{\textcolor[rgb]{0.56,0.35,0.01}{{#1}}}
\newcommand{\AlertTok}[1]{\textcolor[rgb]{0.94,0.16,0.16}{{#1}}}
\newcommand{\FunctionTok}[1]{\textcolor[rgb]{0.00,0.00,0.00}{{#1}}}
\newcommand{\RegionMarkerTok}[1]{{#1}}
\newcommand{\ErrorTok}[1]{\textbf{{#1}}}
\newcommand{\NormalTok}[1]{{#1}}
\ifxetex
  \usepackage[setpagesize=false, % page size defined by xetex
              unicode=false, % unicode breaks when used with xetex
              xetex]{hyperref}
\else
  \usepackage[unicode=true]{hyperref}
\fi
\hypersetup{breaklinks=true,
            bookmarks=true,
            pdfauthor={CJ},
            pdftitle={Appendix of Exponential Distribution and Central Limit Theorem},
            colorlinks=true,
            citecolor=blue,
            urlcolor=blue,
            linkcolor=magenta,
            pdfborder={0 0 0}}
\urlstyle{same}  % don't use monospace font for urls
\setlength{\parindent}{0pt}
\setlength{\parskip}{6pt plus 2pt minus 1pt}
\setlength{\emergencystretch}{3em}  % prevent overfull lines
\setcounter{secnumdepth}{0}

%%% Change title format to be more compact
\usepackage{titling}
\setlength{\droptitle}{-2em}
  \title{Appendix of Exponential Distribution and Central Limit Theorem}
  \pretitle{\vspace{\droptitle}\centering\huge}
  \posttitle{\par}
  \author{CJ}
  \preauthor{\centering\large\emph}
  \postauthor{\par}
  \predate{\centering\large\emph}
  \postdate{\par}
  \date{Thursday, January 08, 2015}




\begin{document}

\maketitle


{
\hypersetup{linkcolor=black}
\setcounter{tocdepth}{2}
\tableofcontents
}
\subsubsection{Simulation}\label{simulation}

Firstly, calculate the theoretical values of exponential mean and
standard deviation. When \texttt{lambda}= 0.2, both mean and standard
deviation equal to \texttt{5}.

\begin{Shaded}
\begin{Highlighting}[]
\NormalTok{lambda <-}\FloatTok{0.2} 

\CommentTok{# therotical values of sd and mean}
\NormalTok{tsd <-}\StringTok{ }\DecValTok{1}\NormalTok{/lambda; tmn <-}\StringTok{ }\DecValTok{1}\NormalTok{/lambda}

\NormalTok{tmn;tsd}
\end{Highlighting}
\end{Shaded}

Function \texttt{exps} generate a vector of size \texttt{n} contains
transformation of \texttt{m} exponentials. For example, here we use
function \texttt{mean}, which returns a vector of means of \texttt{m}
exponetials.

\begin{Shaded}
\begin{Highlighting}[]
\NormalTok{lambda <-}\FloatTok{0.2}

\NormalTok{n<-}\DecValTok{1000} \CommentTok{# Number of simulations simulations}

\NormalTok{m<-}\DecValTok{40} \CommentTok{# Number of  exponentials.}

\NormalTok{exps<-}\StringTok{ }\NormalTok{function(ne,ns,lambda,seed, }\DataTypeTok{FUN=}\NormalTok{mean)\{        }
        \KeywordTok{set.seed}\NormalTok{(seed)}
        \NormalTok{x=}\OtherTok{NULL}
        \NormalTok{for (i in }\DecValTok{1}\NormalTok{:ns) x=}\KeywordTok{c}\NormalTok{(x, }\KeywordTok{FUN}\NormalTok{(}\KeywordTok{rexp}\NormalTok{(ne,lambda))) }
        \KeywordTok{data.frame}\NormalTok{(x)       }
\NormalTok{\} }
\NormalTok{mns <-}\StringTok{ }\KeywordTok{exps}\NormalTok{(}\DecValTok{40}\NormalTok{,}\DecValTok{1000}\NormalTok{,lambda, }\DecValTok{1000}\NormalTok{, mean)}
\NormalTok{vars <-}\StringTok{ }\KeywordTok{exps}\NormalTok{(}\DecValTok{40}\NormalTok{,}\DecValTok{1000}\NormalTok{,lambda,}\DecValTok{1000}\NormalTok{, }\DataTypeTok{FUN=}\NormalTok{var)}
\end{Highlighting}
\end{Shaded}

\subsubsection{Sample Mean versus Theoretical
Mean}\label{sample-mean-versus-theoretical-mean}

Plot the histogram of a thousand of simulated means

\begin{Shaded}
\begin{Highlighting}[]
\KeywordTok{library}\NormalTok{(ggplot2)}

\CommentTok{# average of means}
\NormalTok{smn <-}\StringTok{ }\KeywordTok{mean}\NormalTok{(mns$x); smn}

\CommentTok{# standard error}
\NormalTok{sse <-}\StringTok{ }\KeywordTok{sd}\NormalTok{(mns$x)/}\KeywordTok{sqrt}\NormalTok{(n); sse}

\CommentTok{# plot histogram}
\NormalTok{g1 <-}\StringTok{ }\KeywordTok{ggplot}\NormalTok{(mns, }\KeywordTok{aes}\NormalTok{(}\DataTypeTok{x=}\NormalTok{x)) }

\NormalTok{myhist<-}\StringTok{ }\NormalTok{function(g, bw,title) \{        }
        \NormalTok{hist <-}\StringTok{ }\NormalTok{g+}\StringTok{ }\KeywordTok{geom_histogram}\NormalTok{(}\DataTypeTok{binwidth=}\NormalTok{bw, }\DataTypeTok{colour=}\StringTok{"black"}\NormalTok{, }\DataTypeTok{fill=}\StringTok{"white"}\NormalTok{)+}
\StringTok{                }\KeywordTok{ggtitle}\NormalTok{(title)+}
\StringTok{                }\KeywordTok{theme}\NormalTok{(}\DataTypeTok{plot.title =} \KeywordTok{element_text}\NormalTok{(}\DataTypeTok{lineheight=}\NormalTok{.}\DecValTok{8}\NormalTok{, }\DataTypeTok{face=}\StringTok{"bold"}\NormalTok{))+}
\StringTok{                }\KeywordTok{xlab}\NormalTok{(}\StringTok{""}\NormalTok{)+}
\StringTok{                }\KeywordTok{ylab}\NormalTok{(}\StringTok{"Count"}\NormalTok{)}
        \NormalTok{hist }
\NormalTok{\}}

\NormalTok{hist1 <-}\StringTok{ }\KeywordTok{myhist}\NormalTok{(g1, }\FloatTok{0.1}\NormalTok{, }\DataTypeTok{title=}\StringTok{"Fig1: Sample Mean versus Theoretical Mean"}\NormalTok{)}

\CommentTok{# add annotation }
\NormalTok{notex <-}\StringTok{ }\DecValTok{6}\NormalTok{; notey <-}\StringTok{ }\DecValTok{70}
\NormalTok{hist1+}\StringTok{ }\KeywordTok{geom_vline}\NormalTok{(}\KeywordTok{aes}\NormalTok{(}\DataTypeTok{xintercept=}\NormalTok{smn),}\DataTypeTok{color=}\StringTok{"red"}\NormalTok{, }\DataTypeTok{linetype=}\StringTok{"dashed"}\NormalTok{, }\DataTypeTok{size=}\DecValTok{1}\NormalTok{)+}
\StringTok{        }\KeywordTok{annotate}\NormalTok{(}\StringTok{"text"}\NormalTok{, }\DataTypeTok{x =} \NormalTok{notex, }\DataTypeTok{y =} \NormalTok{notey, }
                 \DataTypeTok{label =} \KeywordTok{paste}\NormalTok{(}\StringTok{"sample mean:"}\NormalTok{, }\KeywordTok{as.character}\NormalTok{(}\KeywordTok{round}\NormalTok{(smn,}\DecValTok{2}\NormalTok{))), }\DataTypeTok{color=}\StringTok{"red"}\NormalTok{)+}
\StringTok{        }\KeywordTok{geom_vline}\NormalTok{(}\KeywordTok{aes}\NormalTok{(}\DataTypeTok{xintercept=}\NormalTok{tmn),}\DataTypeTok{color=}\StringTok{"blue"}\NormalTok{, }\DataTypeTok{linetype=}\StringTok{"dashed"}\NormalTok{, }\DataTypeTok{size=}\DecValTok{1}\NormalTok{)+}
\StringTok{        }\KeywordTok{annotate}\NormalTok{(}\StringTok{"text"}\NormalTok{, }\DataTypeTok{x =} \NormalTok{notex, }\DataTypeTok{y =} \NormalTok{notey}\DecValTok{+5}\NormalTok{, }
                 \DataTypeTok{label =} \KeywordTok{paste}\NormalTok{(}\StringTok{"theoretical mean:"}\NormalTok{,  }\KeywordTok{as.character}\NormalTok{(}\KeywordTok{round}\NormalTok{(tmn,}\DecValTok{2}\NormalTok{))), }\DataTypeTok{color=}\StringTok{"blue"}\NormalTok{)}
\end{Highlighting}
\end{Shaded}

As we can see from \texttt{Fig1}, the center of sample means is very
close to the theoretical mean with a standard error 0.02558013.

\subsubsection{Sample Variance versus Theoretical
Variance}\label{sample-variance-versus-theoretical-variance}

Plot the histogram of a thousand of simulated variances

\begin{Shaded}
\begin{Highlighting}[]
\KeywordTok{library}\NormalTok{(ggplot2)}

\CommentTok{# average of variances}
\NormalTok{mv <-}\StringTok{ }\KeywordTok{mean}\NormalTok{(vars$x);mv}

\CommentTok{# standard error}
\NormalTok{sse <-}\StringTok{ }\KeywordTok{sd}\NormalTok{(vars$x)/}\KeywordTok{sqrt}\NormalTok{(n); sse}

\CommentTok{# plot histogram}
\NormalTok{g2 <-}\StringTok{ }\KeywordTok{ggplot}\NormalTok{(vars, }\KeywordTok{aes}\NormalTok{(}\DataTypeTok{x=}\NormalTok{x)) }

\NormalTok{hist2 <-}\StringTok{ }\KeywordTok{myhist}\NormalTok{(g2, }\DecValTok{1}\NormalTok{, }\DataTypeTok{title=}\StringTok{"Fig2: Sample Variance versus Theoretical Variance"}\NormalTok{)}

\CommentTok{# add annotation }

\NormalTok{notex <-}\StringTok{ }\DecValTok{50}\NormalTok{; notey <-}\StringTok{ }\DecValTok{45}

\NormalTok{hist2+}\StringTok{ }\KeywordTok{geom_vline}\NormalTok{(}\KeywordTok{aes}\NormalTok{(}\DataTypeTok{xintercept=}\NormalTok{mv),}\DataTypeTok{color=}\StringTok{"red"}\NormalTok{, }\DataTypeTok{linetype=}\StringTok{"dashed"}\NormalTok{, }\DataTypeTok{size=}\DecValTok{1}\NormalTok{)+}
\StringTok{        }\KeywordTok{annotate}\NormalTok{(}\StringTok{"text"}\NormalTok{, }\DataTypeTok{x =} \NormalTok{notex, }\DataTypeTok{y =} \NormalTok{notey, }
                 \DataTypeTok{label =} \KeywordTok{paste}\NormalTok{(}\StringTok{"sample variance:"}\NormalTok{, }\KeywordTok{as.character}\NormalTok{(}\KeywordTok{round}\NormalTok{(mv,}\DecValTok{2}\NormalTok{))), }\DataTypeTok{color=}\StringTok{"red"}\NormalTok{)+}
\StringTok{        }\KeywordTok{geom_vline}\NormalTok{(}\KeywordTok{aes}\NormalTok{(}\DataTypeTok{xintercept=}\NormalTok{tsd^}\DecValTok{2}\NormalTok{),}\DataTypeTok{color=}\StringTok{"blue"}\NormalTok{, }\DataTypeTok{linetype=}\StringTok{"dashed"}\NormalTok{, }\DataTypeTok{size=}\DecValTok{1}\NormalTok{)+}
\StringTok{        }\KeywordTok{annotate}\NormalTok{(}\StringTok{"text"}\NormalTok{, }\DataTypeTok{x =} \NormalTok{notex, }\DataTypeTok{y =} \NormalTok{notey}\DecValTok{+5}\NormalTok{, }
                 \DataTypeTok{label =} \KeywordTok{paste}\NormalTok{(}\StringTok{"theoretical variance:"}\NormalTok{,  }\KeywordTok{as.character}\NormalTok{(}\KeywordTok{round}\NormalTok{(tsd^}\DecValTok{2}\NormalTok{,}\DecValTok{2}\NormalTok{))), }\DataTypeTok{color=}\StringTok{"blue"}\NormalTok{)}
\end{Highlighting}
\end{Shaded}

\subsubsection{Nomarlization}\label{nomarlization}

Here we normalize both sample mean and variance and compare with the
standard normal distribution.

\begin{Shaded}
\begin{Highlighting}[]
\NormalTok{normalize <-}\StringTok{ }\NormalTok{function(x, xbar, ste)\{        }
        \NormalTok{(x-xbar)/ste       }
\NormalTok{\} }

\NormalTok{Nmns <-}\StringTok{ }\KeywordTok{as.data.frame}\NormalTok{(}\KeywordTok{normalize}\NormalTok{(mns$x, }\KeywordTok{mean}\NormalTok{(mns$x), }\KeywordTok{sd}\NormalTok{(mns$x))); }\KeywordTok{names}\NormalTok{(Nmns) <-}\StringTok{ "x"}
\NormalTok{Nv <-}\StringTok{ }\KeywordTok{as.data.frame}\NormalTok{(}\KeywordTok{normalize}\NormalTok{(vars,}\KeywordTok{mean}\NormalTok{(vars$x), }\KeywordTok{sd}\NormalTok{(vars$x))); }\KeywordTok{names}\NormalTok{(Nmns) <-}\StringTok{ "x"}
\end{Highlighting}
\end{Shaded}

Plot both distributions of sample mean and variance and overlay with the
standard normal distribution.

\begin{Shaded}
\begin{Highlighting}[]
\KeywordTok{library}\NormalTok{(ggplot2)}
\KeywordTok{require}\NormalTok{(gridExtra)}

\CommentTok{# plot histogram}
\NormalTok{g3 <-}\StringTok{ }\KeywordTok{ggplot}\NormalTok{(Nmns, }\KeywordTok{aes}\NormalTok{(}\DataTypeTok{x=}\NormalTok{x)) }
\NormalTok{g4 <-}\StringTok{ }\KeywordTok{ggplot}\NormalTok{(Nv, }\KeywordTok{aes}\NormalTok{(}\DataTypeTok{x=}\NormalTok{x)) }

\CommentTok{# plot density + standard normal distribution }
\NormalTok{myNormPlot <-}\StringTok{ }\NormalTok{function(fh, title)\{        }
        \NormalTok{g<-}\StringTok{ }\NormalTok{fh+}\StringTok{ }\KeywordTok{geom_histogram}\NormalTok{(}\KeywordTok{aes}\NormalTok{(}\DataTypeTok{y=}\NormalTok{..density..), }\DataTypeTok{binwidth=}\FloatTok{0.2}\NormalTok{, }\DataTypeTok{colour=}\StringTok{"black"}\NormalTok{, }\DataTypeTok{fill=}\StringTok{"white"}\NormalTok{)+}
\StringTok{        }\KeywordTok{ggtitle}\NormalTok{(title)+}
\StringTok{        }\KeywordTok{theme}\NormalTok{(}\DataTypeTok{plot.title =} \KeywordTok{element_text}\NormalTok{(}\DataTypeTok{lineheight=}\NormalTok{.}\DecValTok{8}\NormalTok{, }\DataTypeTok{face=}\StringTok{"bold"}\NormalTok{))+}
\StringTok{        }\KeywordTok{xlab}\NormalTok{(}\StringTok{""}\NormalTok{)+}
\StringTok{        }\KeywordTok{ylab}\NormalTok{(}\StringTok{""}\NormalTok{)+}
\StringTok{        }\KeywordTok{stat_function}\NormalTok{(}\DataTypeTok{fun =} \NormalTok{dnorm, }\DataTypeTok{size =} \DecValTok{1}\NormalTok{)+}
\StringTok{        }\KeywordTok{xlim}\NormalTok{(-}\DecValTok{4}\NormalTok{,}\DecValTok{4}\NormalTok{)}
        \NormalTok{g}
        
\NormalTok{\} }

\NormalTok{g3 <-}\StringTok{ }\KeywordTok{myNormPlot}\NormalTok{(g3,}\StringTok{"(a) Normalized Sample Mean"}\NormalTok{)}
\NormalTok{g4 <-}\StringTok{ }\KeywordTok{myNormPlot}\NormalTok{(g4,}\StringTok{"(b) Normalized Sample Variance"}\NormalTok{)}

\KeywordTok{grid.arrange}\NormalTok{(g3, g4, }\DataTypeTok{ncol=}\DecValTok{2}\NormalTok{, }
             \DataTypeTok{main=}\KeywordTok{textGrob}\NormalTok{(}\StringTok{"Fig3: Comparison with Standard Normal"}\NormalTok{,}
                           \DataTypeTok{gp =} \KeywordTok{gpar}\NormalTok{(}\DataTypeTok{fontsize =} \DecValTok{14}\NormalTok{, }
                                     \DataTypeTok{fontface =} \StringTok{"bold"}\NormalTok{))}
             \NormalTok{)}
\end{Highlighting}
\end{Shaded}

\end{document}
